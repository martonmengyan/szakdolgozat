\Chapter{Koncepció}

\Section{A fejezet célja}

A szimulációs környezet inspirációt vesz már elkészített játékokból, mint például a Pixel Dungeon és Darkest Dungeon
nevű játékokból ismerhető szoba és folyosó kapcsolat. Ahol minden szobát egy folyosó köt össze egy másik tetszőleges szobával.
A Darkest Dungeon-hoz képest a játékmenet rugalmasabb, mivel a szobák bármikor elhagyhatóak és megközelíthetőek, ha engedi azt a pálya jelenlegi belső felépítése.
Azaz, ha nincs útban valami, ami megakadályozza.

Illetve inspirációt vesz még továbbá a For The King nevezetű játékból, ahol a fegyverek adott karakter statiszkákból "scalelődnek", azaz a birtokolt statiszkák függvényében
számolódnak ki az adott értékek figyelembe véve a célpont általt birtokolt statisztikákat is.

A fent említett három játék mind körökre osztott, ahogyan a szimulációs környezetünk is az lesz.

A szimulációs környezetünk viszont egyedi lesz abból a szempontból, hogy az ágensek csoportokra vannak osztva és közös céllal rendelkezve próbálják legyőzni egymást és a játékost.
Illetve hogy nem csak a játékosnak létezik 'win condition'-je, hanem az ágenseknek is, mindkét frakciónak különböző.

\iffalse
A hivatkozások jelentős része ehhez a fejezethez szokott kötődni.
(Egy hivatkozás például így néz ki \cite{coombs1987markup}.)
Itt lehet bemutatni a hasonló alkalmazásokat.
\fi

\Section{Tartalom és felépítés}

\iffalse
A fejezet tartalma témától függően változhat. Az alábbiakat attól függően különböző arányban tartalmazhatják.
\begin{itemize}
\item Irodalomkutatás. Amennyiben a dolgozat egy módszer kidolgozására, kifejlesztésére irányul, akkor itt lehet részletesen végignézni (módszertani vagy időrendi bontásban), hogy az eddigiekben milyen eredmények születtek a témakörben.
\item Technológia. Mivel jellemzően kutatásról vagy szoftverfejlesztésről van szó, ezért annak a jellemző elemeit, technikai részleteit itt kell bemutatni.
Ez tehát egy módszeres bevezetés ahhoz, hogy ha valaki nem jártas a témakörben, akkor tudja, hogy a dolgozat milyen aktuálisan elérhető eredményeket, eszközöket használt fel.
\item Piackutatás. Bizonyos témáknál új termék vagy szolgáltatás kifejlesztése a cél.
Ekkor érdemes annak alaposan utánanézni, hogy aktuálisan milyen eszközök érhetők el a piacon.
Ez szoftverek esetében a hasonló alkalmazások bemutatását, táblázatos formában történő összehasonlítását jelentheti.
Szerepelhetnek képek és észrevételek a viszonyításként bemutatott alkalmazásokhoz.
\item Követelmény specifikáció. Külön szakaszban érdemes részletesen kitérni az elkészítendő alkalmazással kapcsolatos követelményekre.
Ehhez tartozhatnak forgatókönyvek (\textit{scenario}-k).
A szemléletesség kedvéért lehet hozzájuk képernyőkép vázlatokat is készíteni, vagy a használati eseteket más módon szemléltetni.
\end{itemize}
\fi

\Section{Menü}

A játék indításakor vagy megállításakor felugró ablak.

\begin{itemize}
\item Start/Continue:
A játék indításakor a menüben a start-ot látjuk, ha a játékot állítjuk meg, akkor a menüben a continue-t látjuk.
Ezen menüpont kiválasztásával kezdhetjük a játékmenetünket vagy folytathatjuk azt.

\item Save:
Ez a menüpont nem jelenik meg a játék indításakor csak akkor, hogyha játékmenetet állítunk meg.
Ezen menüpont kiválasztásával menthetjük a jelenlegi játékmenetünket.

\item Load:
Ezen menüpont mindig jelen van a menüben, elérhető a játék indításakor és megállításakor is.
Ezen menüpont kiválasztásával tölthetjük be az általunk lementett mentéseket.

\item Options:
Ezen menüpont mindig jelen van a menüben, elérhető a játék indításakor és megállításakor is.
Ezen menüpont kiválasztásával egy új ablakra irányít át minket.
Itt lehetőségünk van hangbeállítások személyre szabására, mint például a zene és a játékbeli hangok. (Több csúszka segítségével)
Lehetőségünk van fényerő beállításokra is, mint például a Gamma állítására. (Egy csúszka segítségével)
Nyelvi beállítások a menüre vonatkozóan. (Legördülő fülből kiválasztva)
Vér ki/bekacsolása játék közben. (Egy ki/bepipálható rublika)
Végezetül egy visszagomb ami visszairányít minket a menübe.

\item Log out:
Kiléptet minket a jelenlegi fiókunkból, visszairányítva a bejelentkezési ablakhoz.
A log-out a menüben mindig szerepel.

\item Main menu:
Ez a gomb akkor szerepel a menüben hogyha egy játékmenet megállításával kerülünk a menübe.
Ez befejezi a jelenlegi játékmenetünket és visszaírányít minket a főmenübe.
\end{itemize}

Ahol a Fekete háttér, az a megállított játék képe lenne elhomályosítva.
Ha még nem kezdtünk bele egy játékmenetbe, akkor pedig általunk megadott játékmenet képernyőmentése lenne a háttérben elhomályosítva.

\Section{Szimulációs környezet specifikálása}

\subsection{Map}

A map szobák és folyosók összesége, ahol a szobák alakjánál törekszünk a négyzet alakú szobák elkerülésére. A folyosók általában rövidek és egy szóbából akár több is nyílik egy adott szobára.
A szobákban és a folyosókban bejárható terület egy mátrixhoz hasonló, ahol a celláit blokkoknak nevezzük.
Minden blokkhoz tartoznak adatok.

Ezek az adatok:

\begin{itemize}
    \item X koordináta (egészérték)
    \item Y koordináta (egészérték)
    \item Entity (intiger típusú érték,  az adott blokkon lévő entity ID-ját tartalmazza)
    \item Tárgy (integer típusú tömb, az adott blokkon lévő tárgyak ID-ját tartalmazza)
    \item Effekt (integer típusú tömb, az adott blokkon lévő effektek ID-ját tartalmazza)
    \item Fény (integer típusú tömb, az adott cella megvilágítási értékeit vizsgálja)
\end{itemize}

Fontosabb kizáró feltételek:

\begin{itemize}
    \item Ajtó csak szoba és a folyosó, vagy szoba és a szoba között létezzen.
    \item Folyosóból ne nyíljon ajtó folyosóra.
    \item Két különböző folyósó ne érjen össze.
    \item Minden szobából nyíljon ajtó, úgy hogy lehetséges legyen elhagyni.
    \item A map bármelyik részéből elérhető legyen a map bármelyik tetszőleges helye.
\end{itemize}

\subsection{Szobák és folyosók}

Minden szomszédos szobát és folyósót egy ajtó választ el egymástól.
Celláin korlátozottan helyezkedhetnek el térbeli objektumok(Lásd: oszlop), amely a cellát nem bejárható cellává alakít és korlátozza a látást.

\Section{Karakterünk}

\subsection{Szint}

Karakterünk rendelkezik egy szinttel, ami a legelején az 1es szintről indul.
Entity legyőzése ad karakterünknek tapasztalatpontot, amelyből ha megszerzünk egy adott mennyiséget, a karaktünk szintet lép.
Minden következő szinthez 10\%-kal több tapasztalatpont szükséges, mint az előző szinthez.
Szintlépéskor lehetőségünk van a Karakter oldalon a szintlépéskor kapott 3 pontunkat elosztani a fő 4 statisztikára.(Lásd: Statisztika)

\subsection{Statisztika}

Entityk és karakterünk 4 alap statisztikával rendelkezik.
Ez a 4 statisztika a következők: Erő, Kitérés, Kitartás, Pontosság.

Kezdő statisztika:
\begin{itemize}
    \item Erő: 10
    \item Kitérés: 10
    \item Kitartás: 10
    \item Pontosság: 10
\end{itemize}

\Section{Ágensek}

\subsection{Tolvaj}

Kezdő statisztika:
\begin{itemize}
    \item Erő: 8
    \item Kitérés: 12
    \item Kitartás: 8
    \item Pontosság: 12
\end{itemize}

\subsection{Harcos}

Kezdő statisztika:
\begin{itemize}
    \item Erő: 12
    \item Kitérés: 8
    \item Kitartás: 12
    \item Pontosság: 8
\end{itemize}

\subsection{Sebzés számlálás}

...

\subsection{Kommunikáció közöttük}

...

\subsection{Megfigyelési akcióik}

...

\subsection{Elérendő céljaik}

...

\Section{Combat}

A Játékban nincs igazi combat fázis, combat esetén a játék nem szakítódik meg, hanem minden folytatódik ahogy folytatódna.
Ez azt jelenti, hogy hiába vagyunk egy karakterrel harcban, minden más karakter elvégzi a saját cselekvését, minden cselekvésünk után.


\Section{Tárgyak}

A játékban elhelyezett tárgyak, a cellákon létezhetnek vagy map generálás után,
vagy pedig utólag is kerülhet oda általunk, vagy más Entity által. (Lásd: Entity a halálakor eldob egy tárgyat)
A cellákon létező tárgyak nem korlátozzák a karakterek mozgását.
Felvételük csak az adott cellára lépve külön akcióval tehető meg.

\subsection{Kulcs}

Vagy Entityktől, vagy celláról szerezhetőek meg.
Szerepük a ládák kinyitása a játékos által.

\subsection{Entity Helm no.1}

Entity elpusztulásakor az adott cellán hagyja hátra 10\%-os eséllyel, de egyszerre csak egy tárgyat hagyhat hátra.
Játékos által használható felszerelési tárgy, amely pozitív és negatív statisztikát ad, amelyek hozzáadódnak, illetve kivonódnak a játékos jelenlegi statisztikáiból.

\begin{itemize}
    \item Erő: 2
    \item Kitérés: -2
    \item Kitartás: 2
    \item Pontosság: -2
\end{itemize}

\subsection{Entity Helm no.2}

Entity elpusztulásakor az adott cellán hagyja hátra 10\%-os eséllyel, de egyszerre csak egy tárgyat hagyhat hátra.
Játékos által használható felszerelési tárgy, amely pozitív és negatív statisztikát ad, amelyek hozzáadódnak, illetve kivonódnak a játékos jelenlegi statisztikáiból.

\begin{itemize}
    \item Erő: -2
    \item Kitérés: 2
    \item Kitartás: -2
    \item Pontosság: 2
\end{itemize}

\subsection{Entity Armor no.1}

Entity elpusztulásakor az adott cellán hagyja hátra 10\%-os eséllyel, de egyszerre csak egy tárgyat hagyhat hátra.
Játékos által használható felszerelési tárgy, amely pozitív és negatív statisztikát ad, amelyek hozzáadódnak, illetve kivonódnak a játékos jelenlegi statisztikáiból.

\begin{itemize}
    \item Erő: 2
    \item Kitérés: -2
    \item Kitartás: 2
    \item Pontosság: -2
\end{itemize}

\subsection{Entity Armor no.2}

Entity elpusztulásakor az adott cellán hagyja hátra 10\%-os eséllyel, de egyszerre csak egy tárgyat hagyhat hátra.
Játékos által használható felszerelési tárgy, amely pozitív és negatív statisztikát ad, amelyek hozzáadódnak, illetve kivonódnak a játékos jelenlegi statisztikáiból.

\begin{itemize}
    \item Erő: -2
    \item Kitérés: 2
    \item Kitartás: -2
    \item Pontosság: 2
\end{itemize}

\subsection{Entity Weapon no.1}

Entity elpusztulásakor az adott cellán hagyja hátra 10\%-os eséllyel, de egyszerre csak egy tárgyat hagyhat hátra.
Játékos által használható felszerelési tárgy, amely pozitív és negatív statisztikát ad, amelyek hozzáadódnak, illetve kivonódnak a játékos jelenlegi statisztikáiból.

\begin{itemize}
    \item Erő: 3
    \item Kitérés: -1
    \item Kitartás: 0
    \item Pontosság: 1
\end{itemize}

\subsection{Entity Weapon no.2}

Entity elpusztulásakor az adott cellán hagyja hátra 10\%-os eséllyel, de egyszerre csak egy tárgyat hagyhat hátra.
Játékos által használható felszerelési tárgy, amely pozitív és negatív statisztikát ad, amelyek hozzáadódnak, illetve kivonódnak a játékos jelenlegi statisztikáiból.

\begin{itemize}
    \item Erő: 1
    \item Kitérés: 1
    \item Kitartás: 0
    \item Pontosság: 3
\end{itemize}

\Section{Mapon található mozgást kizáró objektumok}

\subsection{Láda}

A ládák map generáláskor kerülnek bizonyos cellákra.
Entityk dobják a kulcsukat, vagy a mapra generálódik le valamelyik cellára.
Ládát tartalmazó cellára nem lehetséges a mozgás.
Interakcióval kinyitható, ha az Inventory tartalmazza a kulcsát. (Lásd: Interface)

\subsection{Oszlop}

Az oszlopok map generáláskor kerülnek bizonyos cellákra.
Oszlopoknak semmilyen különleges funkciója nincs azonkívül, hogy map komplexitását kívánja növelni azzal,
hogy mozgást és látást korlátozó szerepet lát el.

\subsection{Zárt ajtó}

Az ajtók map generáláskor kerülnek csak olyan cellákra, ahol egy szobából egy folyosón lehetséges elhagyni a szobát.
Mindig zárt alakban kerül az adott cellákra, kinyitásához nem szükséges semmilyen különleges tárgy. Ha bármely karakter egy ajtót tartalmazó cellára lép, az ajtó kinyílik amíg Entityt tartalmaz az adott cella.
Ahogyan az Oszlopok, a zárt ajtók is map komplexitását kívánja növelni azzal,hogy látást korlátozó szerepet lát el.

\Section{Idő}

Az idő a játékon belül minden általunk végrehajtott akció által telik.
Ha elvégzünk egy akciót, akkor az irányítás átkerül egy másik Entityre,
ami végig megy minden Entity, és ha végig ment a sor, akkor újra mi következünk.

\Section{Kamera}

A kamera normál esetben a karakterünk felett helyezkedik el, és ezáltal tekinthetünk le a map.
A kamera használható korlátozottan szabad módban is, ahol a kamera elrugaszkodik a karakterünktől, és körbe nézhetünk.

\Section{Láthatóság}

Az általunk nem felfedezett map részek, semmilyen módon nem láthatóak általunk, ha már egyszer bejártuk az adott területet, 
de nem látja a karakterünk, akkor csak a terület alakját láthatjuk, az adott cellák frissülését(Lásd: Új Entity megjelenése) már nem fogjuk látni.

\Section{Irányítás}

\begin{itemize}
    \item Kamera szabad mód be- és kikapcsolása: SPACE
    \item Kamera szabadmódbna mozgatása: WASD
    \item A játéktérben karakterünk mozgásra utasítása: LEFT CLICK (általunk már felfedezett cellára)
    \item A Menü megnyitása: ESCAPE
    \item Az Inventory megnyitása: I
    \item Map megnyitása: M
    \item Karakter oldal megnyitása: K
\end{itemize}

\Section{Interface felület}

\subsection{Menü}

\begin{itemize}
    \item prekondíció: A szimulációs ablakban vagyunk.
    \item általános működés: Rávisszük az egeret, majd bal kattintással megnyitjuk.
    \item alternatív esetek: Rossz menü ablak nyílik meg.
    \item Postkondíció: Megnyílt a menü ablak.
    \item kivételes esetek: Nem nyílt meg az ablak.
\end{itemize}

\subsection{Inventory}

\begin{itemize}
    \item prekondíció: A szimulációs ablakban vagyunk.
    \item általános működés: Rávisszük az egeret, majd bal kattintással megnyitjuk.
    \item alternatív esetek: Rossz ablak nyílik meg.
    \item Postkondíció: Megnyílt az inventory ablak.
    \item kivételes esetek: Nem nyílt meg az ablak.
\end{itemize}

\subsection{Map}

\begin{itemize}
    \item prekondíció: A szimulációs ablakban vagyunk.
    \item általános működés: Rávisszük az egeret, majd bal kattintással megnyitjuk.
    \item alternatív esetek: Rossz ablak nyílik meg.
    \item Postkondíció: Megnyílt a Map ablak.
    \item kivételes esetek: Nem nyílt meg az ablak.
\end{itemize}

\subsection{Karakter}

\begin{itemize}
    \item prekondíció: A szimulációs ablakban vagyunk.
    \item általános működés: Rávisszük az egeret, majd bal kattintással megnyitjuk.
    \item alternatív esetek: Rossz ablak nyílik meg.
    \item Postkondíció: Megnyílt a Karakter ablak.
    \item kivételes esetek: Nem nyílt meg az ablak.
\end{itemize}

\subsection{Dobás}

\begin{itemize}
    \item prekondíció: A szimulációs ablakban vagyunk.
    \item általános működés: Rávisszük az egeret, majd bal kattintás után kiválasztjuk a dobandó Tárgyat, és a cél cellát.
    \item alternatív esetek: Rossz tárgyat dob el.
    \item Postkondíció: Eldobtuk az adott tárgyat a célbavett cellára.
    \item kivételes esetek: Nem dobta el a Tárgyat.
\end{itemize}

\subsection{Felvétel}

\begin{itemize}
    \item prekondíció: A szimulációs ablakban vagyunk és a karakterünk celláján létezik valamilyen tárgy.
    \item általános működés: Rávisszük az egeret, majd bal kattintással felvesszük a Tárgyat, ha kiválaszottuk melyik Tárgyat vesszük fel a bal és jobbra mutató nyilak segítségével.
    \item alternatív esetek: Rossz Tárgyat vesz fel.
    \item Postkondíció: Felvettük a kiválasztott Tárgyat.
    \item kivételes esetek: Nem vette fel a Tárgyat.
\end{itemize}

\subsection{Körbetekintés}

\begin{itemize}
    \item prekondíció: A szimulációs ablakban vagyunk.
    \item általános működés: Rávisszük az egeret, majd bal kattintással elindítjuk a folyamatot.
    \item alternatív esetek: Nem kapunk rálátást a körbeforgott területre.
    \item Postkondíció: Rálátást nyertünk a környezetünkre.
    \item kivételes esetek: Nem fordult körbe a karakter.
\end{itemize}

\subsection{Láda nyitás}

\begin{itemize}
    \item prekondíció: Ládát tartalmazó cellához vízszintesen vagy függőlegesen közvetlen szomszédos cellán tartózkodunk.
    \item általános működés: Rávisszük az egeret a felbukkanó ikonra, majd bal kattintással elindítjuk a folyamatot.
    \item alternatív esetek: Nem tartalmaz semmit a láda.
    \item Postkondíció: Kinyílt a láda, az adott tárgyat megkaptuk.
    \item kivételes esetek: Nem nyílt ki a láda.
\end{itemize}