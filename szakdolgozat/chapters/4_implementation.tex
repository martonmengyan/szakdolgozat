\Chapter{Megvalósítás}

\Section{Felhasznált technológiák}

\SubSection{Java}

A Java általános célú, objektumorientált programozási nyelv, amelyet a Sun Microsystems fejlesztett a 90-es évek elejétől kezdve egészen 2009-ig, amikor a céget felvásárolta az Oracle \cite{arnold2005java}.

\SubSection{Java Swing}

A Swing osztályok kiküszöbölik a Java legnagyobb gyengeségét, a viszonylag primitív felhasználói felület eszközkészletét. A Java Swing segít a Swing osztályok teljes előnyeinek kihasználásában,
részletes leírást adva minden osztályról és felületről a kulcsfontosságú Swing csomagokban \cite{10.5555/291162}.

Más, hasonló szerkesztő és szimulációs rendszerek esetében is népszerű választás. Ez köszönhető egyrészt a Java nyelv elterjedtségének, másrészt, hogy napjainkban is az egyik legegyszerűbben átvihető \textit{widget toolkit}-ról van szó.

\Section{Az ágensek vezérlése}

Ebben a szakaszban a ágens heurisztikus mozgását és akcióit végrehajtó biztosító függvények bemutatása történik meg röviden.

\section{wantToAttack}

\begin{java}
if (wantToAttack) {
    attackAction(gp.entityList, isEnemyCloseIndexList);
}    
\end{java}

Boolean típusú változó, amely akkor igaz, ha közvetlen szomszédos blockokon létező
ellenséges ágensek száma nagyobb, mint 0.

Igaz érték esetén meghívódik az attackAction() függvény, amelynek paramétere az ágensekből álló lista és 
az isEnemyCloseIndexList, amely a közeli blockon lévő ellenséges ágensek közül a legkevesebb
HP-val rendelkező ágens indexét adja vissza az ágens listából.

Az attackAction() függvény beállítja az adott ágens irányát, arra amelyik irányba hajta végre a támadását és 
elvégzi a támadást.

Támadás előtt először megvizsgálja betalált-e a találat, ha betalált, akkor meghívódik a damaga() függvény, amely
kivonja az adott sebzést az adott ágenstől, amelyik elszenvedte a támadást.

\section{wantToEquip}

\begin{java}
else if (wantToEquip) {			
    equipBetterItem(
        inventoryList.get(inventoryList.size() - 1));
}
\end{java}

Boolean típusú változó, amely akkor igaz, ha az adott ágens inventoryjában az utoljára felvett tárgy hordható
típusú és összességében több statisztikát ad, mint az ágens által jelenleg hordott azonos típusú tárgy.

Az equipBetterItem() függvénynek egy paramétere van, amely egy Integer szám, ami az adott ágens inventoryjának utolsó
használt slotja, amely a legutoljára felvett tárgy. Meghívásakor leszereli az eddig hordott tárgyat, és felszereli az újonnan felvett tárgyat.
Statisztikák ablakon nyomon lehet követni ezt a változást.

\section{wantToPickUp}

\begin{java}
else if (wantToPickUp) {
    pickUpObject(curObjIndex);
}
\end{java}

Boolean típusú változó, amely akkor igaz, ha létezik valamilyen tárgy az ágens jelenlegi blockján.

Ha igaz értéket ad vissza, akkor meghívódik a pickUpObject() függvény, amelynek paramétere az adott blockon lévő tárgy
indexe a tárgyak listájában. Meghívásakor az adott tárgy bekerül az ágens első nem használt inventory slotjába.

\section{wantToDeleteAnItem}

\begin{java}
else if (wantToDeleteAnItem) {
    System.out.println("Deleted item: "
        + inventoryList.get(deleteIndex).name);
    inventoryList.remove(deleteIndex);
}
\end{java}

Boolean típusú változó, amely akkor igaz, ha létezik valamilyen nem viselt tárgy az ágens inventoryjában.

Igaz érték esetén megtörténik a deletedIndex által tárolt elsőnek talált nem használt tárgy indexének a törlése az inventoryjából, 
amely nem az ágens által hordott és nem kulcs.

\section{wantToOpenDoor}
\begin{java}
else if (wantToOpenDoor) {
    openDoor(doorIndex);
}
\end{java}

Boolean típusú változó, amely akkor igaz, ha létezik ajtó az ágens szomszédos blockjában és van az ágens inventoryjában legalább 1 kulcs.

Az openDoor() függvénynek egy paramétere van, amely az ajtó indexe a tárgy listában. Meghívásakor
az ajtó objektum eltűnik, és eltávolítja a felhasznált kulcsot az ágens inventoryjából.

\section{wantToMoveToItem}

\begin{java}
else if (wantToMoveToItem) {
    if (cannotMove == false) {
        moveToItemBlock(gp.objectList,
        isItemCloseIndexList);
        entityMove();
    } else System.out.println("Skipped Turn");
}
\end{java}


Boolean típusú változó, amely akkor igaz, ha a közeli blockokon lévő tárgyak listája nagyobb, mint 0 és az ágens inventoryja nincs tele.

Valódi ágens mozgás csak akkor jön létre, ha van olyan szomszédos block, amelyre képes lépni, ezt a CannotMove false értéke biztosítja.

Majd meghívódik a moveToItemBlock(), amelynek két paramétere van, az objektum lista és a szomszédos közelében lévő tárgyak indexének a listája.
Ez a függvény beállítja az ágens direction értéket.

Majd meghívódik az entityMove() függvény, amely az ágens által választott irányba lép előre egyet és vagy felrakja a hashMap-re, vagy növeli az értékét.

\section{wantToMove}

\begin{java}
else if (wantToMove) {
    if (cannotMove == false) {
        ArrayList<Point> detectableBlocks =
            detectArray(gp.entityList, gp.objectList);
        if (!checkIfNewEnemy(detectableBlocks,
        gp.entityList)) {
            if (!checkIfNewItem(detectableBlocks,
            gp.objectList)) {
                possibleBlocks();
            }
        }
        entityMove();
    } else System.out.println("Skipped Turn");
}
\end{java}

Boolean típusú változó, amely mindig igaz, ha minden fentebb sorolt boolean változú hamis, ebben a pontban összefoglalt akció fog megtörténni.

Valódi ágens mozgás csak akkor jön létre, ha van olyan szomszédos block, amelyre képes lépni, ezt a CannotMove false értéke biztosítja.

A pontokat tároló listában a detectArray() függvény minden olyan x,y párost átad, amely az ágenstől vízszintes vagy függőlegesen 2 blocknyira van
és tartalmaz valamilyen kívánt tárgyat. A detectArray két paramétere az ágensek listája és az objektum lista.

A checkIfNewEnemy() függvény paramétere a pontokat tároló lista és az ágensek listája. Ha talál valamilyen ellenséges ágenst
a vizsgálandó pontokon, akkor azt az irányt fogja beállítani az ágensnek, amely az ellenséges ágens felé néz.

Utána meghívódik az entityMove() függvény, amely a kiválasztott irányba lép egyet és vagy felrakja a hashMap-re, vagy növeli az értékét.

Ha a checkIfNewEnemy() hamis értéket ad vissza, akkor megvizsgálja a checkIfNewItem() függvényt, amelynek két paramétere van, a
vizsgálandó pontokat tároló lista és az objektumok listája.

Ha talál valamilyen számára érdekes tárgyat a vizsgálandó pontokon, akkor azt az irányt fogja beállítani az ágensnek, amely a kívánt tárgy felé néz.

Ha a checkIfNewItem() függvény is hamis értéket ad vissza, akkor meghívódik a possibleBlocks() függvény, amely felel a map felfedezéséért.

A függvény megvizsgálja a lehetséges lépéseket, hogy van-e köztük olyan block, amelyen még egyszer sem járt az ágens. Ha több ilyen van, akkor
véletlenszerűen választ egyet azok közül.

Ha már minden lehetséges blockon járt legalább egyszer, akkor a lehetséges lépések közül a legkevesebbszer bejárt blockot választja. Ha 
több ilyen block is van, amin ugyanannyiszor volt, akkor véletlenszerűen választ közülük egyet.

\Section{Java Swing GUI fontosabb részeinek bemutatása}

JFrame: ez az alapja a Swing alkalmazásoknak.

\medskip

\noindent Nehány fontosabb függvénye a JFrame-nek:
\begin{itemize}
    \item void add (c komponens): Hozzáad egy komponenst.
    \item void setSize (int szélesség, int magasság): Egy komponens szélességének és magasságának a beállítása.
    \item void setVisible (logikai érték a): Ha \textit{a} igaz, akkor a komponens megjelenik a képernyőn.
    \item void setResizable(logikai érték a): Ha \textit{a} igaz, akkor az ablak nem újraméretezhető.
    \item void pack(): úgy méretezi a keretet, hogy annak minden tartalma elérje vagy meghaladja a kívánt méretet.
    \item void setDefaultCloseOperation(): a következő paraméterekkel a következő tevékenységek mennek végbe bezáráskor:
    \begin{itemize}
        \item[1.] JFrame.EXIT\_ON\_CLOSE: Bezárja az alkalmazást.
        \item[2.] JFrame.HIDE\_ON\_CLOSE: Keret elrejtése, de az alkalmazás tovább fut.
        \item[3.] JFrame.DISPOSE\_ON\_CLOSE: Keretobjektum eldobása, de az alkalmazás tovább fut.
        \item[4.] JFrame.DO\_NOTHING\_ON\_CLOSE: Nem csinál semmit.
    \end{itemize}
\end{itemize}
