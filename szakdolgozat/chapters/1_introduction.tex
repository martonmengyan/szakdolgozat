\Chapter{Bevezetés}

A szakdolgozat egy szimulációs környezet létrehozásával fog kezdődni, amelyet felül nézetből vehetünk szemügyre.
Körökre osztott, az ágenseket egy fix sorrendbe rakjuk, amely meghatározza, mikor jön melyik ágens.
A szimulációs környezet véges számú különböző szobákból és folyósókból épül fel. Két szobát egy folyosó köt össze, amelyet a folyosó mindkét végén egy ajtó zár le.
Ezeket a részeket 'szint'-eknek nevezzük(?fix pályák vagy valamennyire randomizált pályák?). Minden szint különböző terep nehézségeket okoz. Pl. lassabb mozgás. 
Ahhoz hogy a következő szintre léphessünk tovább, teljesíteni kell az adott szintet először.

Az ágensek két nagyobb csoportra vannak osztva, amelyeket 'frakció'-knak nevezünk. Minden frakciónak van egy teljesítendő célja, amely teljesítésénél véget ér a szimuláció.
Az ágensek miközben keresik a célukat, figyelembe veszik az útközben észrevett tárgyakat, kulcsokat, ajtókat és ellenséges ágenseket és ezek alapján tervezik meg következő lépésüket.

A szobák és folyosók blokkokból épülnek fel. Ezeken a blokkokon helyezkedhetnek el a karakterek limitáltan, azaz egynél több karakter nem tartózkodhat rajta.
