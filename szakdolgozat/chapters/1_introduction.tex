\Chapter{Bevezetés}

A szakdolgozat egy szimulációs környezet létrehozásával fog kezdődni, amelyet felül nézetből vehetünk szemügyre.
Körökre osztott, az ágenseket egy fix sorrendbe rakjuk, amely meghatározza, mikor jön melyik ágens.
A szimulációs környezet véges számú különböző szobákból és folyósókból épül fel. Két szobát egy folyosó köt össze, amelyet a folyosó mindkét végén egy ajtó zár le. 

A szobák és folyosók blokkokból épülnek fel. Ezeken a blokkokon helyezkedhetnek el a karakterek limitáltan, azaz egynél több karakter nem tartózkodhat rajta.

Az ágensek két nagyobb csoportra vannak osztva, amelyeket \textit{frakció}-knak nevezünk. Minden \textit{frakciónak} van egy teljesítendő célja, amely teljesítésénél véget ér a szimuláció.
Az ágensek miközben keresik a célukat, figyelembe veszik az útközben észrevett tárgyakat, kulcsokat, ajtókat és ellenséges ágenseket és ezek alapján tervezik meg következő lépésüket.

Ha az ágensek \textit{Inventory} mérete nem telített, akkor az észrevett tárgyakat megközelítik és felveszik.

Az \textit{Inventory}-ba bekerült hordható tárgyakat, ha nincs semmilyen más nagyobb prioritású elintézni valójuk, akkor megvizsgálják annak statisztikáit és összehasonlítják a vele azonos típusú \textit{typeName} változójú általuk hordott tárgy statisztikáival, ezáltal meghatározva, hogy jobb-e vagy sem.

Mivel az \textit{Inventory} véges ezért valamilyen módon kezelniük kell a saját \textit{Inventory} helyeiket, hogy ne teljen meg értelmetlenül.

Ha van ellenséges ágens vagy tárgy az észlelési zónájukban, akkor azokat megközelítik. Több ágens észlevételénél a legkevesebb életerővel  rendelkezőt fogja támadni.

Ezen kívül megvalósításra kerül a legfontosabb része a szimulációnak, amely a \textit{map} herusztikus felfedezése.

A program Java programozási nyelven \cite{arnold2005java} készül, a \textit{Java Swing} csomag \cite{10.5555/291162} használatával.