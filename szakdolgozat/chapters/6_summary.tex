\Chapter{Összefoglalás}

A félévben a szakdolgozaton dolgozva sikerült felélénkíteni az egyetem alatt megtanult java ismereteim,
és bővíteni azokat a java Swing-gel. Emellett nagy segítséget nyújtott a Szoftvertechológiák nevezetű tárgyon
tanult ismeretek, amelyekben megtanultuk, hogy hogyan tervezzünk meg egy jövőben elkészülő alkalmazást.
A félévben kisebb nehezségekbe ütköztem a tervezés és implementáció során, amelyek megoldása fejlesztette látásmódom,
így ha legközelebb valamilyen hasonló feladatom lenne képes lennék hatékonyabban megoldani azt.

A tesztek során néhány ágens látványosan kevés utat tett meg a szimulációban, ennek oka az, ha összetalálkozott egy másik
ellentétes ágenssel, akkor valamelyik ágens a számolások után elpusztult.

A szimuláció futási ideje nagyban függ attól, hogy mennyire segítjük az ágenseket azzal, hogy teszünk-e több kulcsot a pályára, mint
amennyi ajtó van. Hiszen, ha ugyanannyi kulcs van a pályán, mint ajtó és feltételezzük, hogy olyan helyekre vannak eltéve, ahol
nem lehetséges az, hogy úgy használják el minden kulcsukat, hogy ne férjenek hozzá további kulcsokhoz. Ebben az esetben annak az ágensnek, akinek elfogyott
a potenciálisan megszerezhető kulcsok száma, akkor a másik ágens útvonalát kell megkeresni és követnie azt a tovább haladáshoz.

Illetve lehet növelni a tesztek komplexitását azzal is, hogy az ágensek másodlagos céljait többször vettjük vizsgálat alá. Azaz a szimulációba több tárgy észlelést, vizsgálatot és \textit{combat}-ot, sebzés számolást teszünk be.